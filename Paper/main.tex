\documentclass[11pt]{article}

% !TEX root = main.tex

% Change "review" to "final" to generate the final (sometimes called camera-ready) version.
% Change to "preprint" to generate a non-anonymous version with page numbers.
\usepackage[preprint]{acl}

% Standard package includes
\usepackage{times}
\usepackage{latexsym}

% For proper rendering and hyphenation of words containing Latin characters (including in bib files)
\usepackage[T1]{fontenc}
% For Vietnamese characters
% \usepackage[T5]{fontenc}
% See https://www.latex-project.org/help/documentation/encguide.pdf for other character sets

% This assumes your files are encoded as UTF8
\usepackage[utf8]{inputenc}

% This is not strictly necessary, and may be commented out,
% but it will improve the layout of the manuscript,
% and will typically save some space.
\usepackage{microtype}

% This is also not strictly necessary, and may be commented out.
% However, it will improve the aesthetics of text in
% the typewriter font.
\usepackage{inconsolata}
\usepackage{fvextra}

\usepackage{xcolor}
\usepackage{fancyvrb}

\usepackage{xparse}
\usepackage[most]{tcolorbox}
\usepackage{enumitem}

\definecolor{uibkblue}{cmyk}{1,0.6,0,0.65}%
\definecolor{uibkbluel}{rgb}{0.89,0.94,1.00}%

% --- Styling (adjust colors/fonts as you like) ---
\tcbset{
  qex/.style={
    enhanced,
    boxrule=0.8pt,
    arc=3pt,
    left=6pt,right=6pt,top=6pt,bottom=6pt,
    colback=uibkbluel!3,
    colframe=uibkblue,
    fonttitle=\bfseries,
    title=Question example
  }
}

% --- Field macros (used inside the environment) ---
\NewDocumentCommand{\question}{m}{%
  \textbf{Question:} #1\par
}
\NewDocumentCommand{\context}{m}{%
  \textbf{Context:} #1\par
}
\NewDocumentCommand{\recencylabel}{m}{%
  \textbf{Recency label:} #1\par
}

% Two-context / two-label helpers
\NewDocumentCommand{\contextii}{m m}{%
  \textbf{Context 1:} #1\par
  \textbf{Context 2:} #2\par
}
\NewDocumentCommand{\labelsii}{m m}{%
  \textbf{Recency label 1:} #1\par
  \textbf{Recency label 2:} #2\par
}

% Optional: a small separator between sections (useful when you mix fields)
\NewDocumentCommand{\qexsep}{}{%
  \vspace{0.3em}\hrule\vspace{0.5em}
}

% --- The environment itself ---
\NewDocumentEnvironment{questionexample}{ O{} }{%
  \begin{tcolorbox}[qex,#1]%
}{%
  \end{tcolorbox}%
}

% Define a stylish prompt environment
\definecolor{promptbg}{RGB}{245,245,245} % light gray background
\DefineVerbatimEnvironment{promptblock}{Verbatim}{
  breaklines=true,
  breakanywhere=true,
  fontsize=\small,
  frame=single,                 % single-line border
  rulecolor=\color{gray},       % border color
  framerule=0.2pt,              % thinner border
  framesep=2pt,                 % reduce padding so it doesn't look like a double frame
  numbers=left,                 % line numbers (optional)
  numbersep=5pt,
  baselinestretch=1.1,
  xleftmargin=5pt,
  xrightmargin=5pt,
  fillcolor=\color{white},      % background color
  commandchars=\\\{\}           % allows LaTeX commands inside
}

%Including images in your LaTeX document requires adding
%additional package(s)
\usepackage{graphicx}

\usepackage{listing}

% For enquote
\usepackage{csquotes}

\usepackage{siunitx}
\usepackage{dblfloatfix}  
\usepackage{placeins}

\usepackage{booktabs} % Required for \toprule, \midrule, \bottomrule
\usepackage{amssymb}  % Required for \checkmark
\usepackage{pifont}   % Required for \ding (nicer checks and crosses)

% Define convenient macros for check and cross
\newcommand{\cmark}{\ding{51}} % Checkmark
\newcommand{\xmark}{\ding{55}} % Cross (X)

% For JSON in appendix
\usepackage{listings}
\usepackage{bera}

\colorlet{punct}{red!60!black}
\definecolor{background}{HTML}{EEEEEE}
\definecolor{delim}{RGB}{20,105,176}
\colorlet{numb}{magenta!60!black}

\lstdefinelanguage{JSON}{
    basicstyle=\normalfont\ttfamily,
    numbers=left,
    numberstyle=\scriptsize,
    stepnumber=1,
    numbersep=8pt,
    showstringspaces=false,
    breaklines=true,
    frame=lines,
    backgroundcolor=\color{background},
    literate=
     *{0}{{{\color{numb}0}}}{1}
      {1}{{{\color{numb}1}}}{1}
      {2}{{{\color{numb}2}}}{1}
      {3}{{{\color{numb}3}}}{1}
      {4}{{{\color{numb}4}}}{1}
      {5}{{{\color{numb}5}}}{1}
      {6}{{{\color{numb}6}}}{1}
      {7}{{{\color{numb}7}}}{1}
      {8}{{{\color{numb}8}}}{1}
      {9}{{{\color{numb}9}}}{1}
      {:}{{{\color{punct}{:}}}}{1}
      {,}{{{\color{punct}{,}}}}{1}
      {\{}{{{\color{delim}{\{}}}}{1}
      {\}}{{{\color{delim}{\}}}}}{1}
      {[}{{{\color{delim}{[}}}}{1}
      {]}{{{\color{delim}{]}}}}{1},
}

% If the title and author information does not fit in the area allocated, uncomment the following
%
%\setlength\titlebox{<dim>}
%
% and set <dim> to something 5cm or larger.
\title{When Do Answers Change? Estimating Question Recency Demands in QA with Multi-Events}

% Author information can be set in various styles:
% For several authors from the same institution:
% \author{Author 1 \and ... \and Author n \\
%         Address line \\ ... \\ Address line}
% if the names do not fit well on one line use
%         Author 1 \\ {\bf Author 2} \\ ... \\ {\bf Author n} \\
% For authors from different institutions:
% \author{Author 1 \\ Address line \\  ... \\ Address line
%         \And  ... \And
%         Author n \\ Address line \\ ... \\ Address line}
% To start a separate ``row'' of authors use \AND, as in
% \author{Author 1 \\ Address line \\  ... \\ Address line
%         \AND
%         Author 2 \\ Address line \\ ... \\ Address line \And
%         Author 3 \\ Address line \\ ... \\ Address line}

\author{Peter Schulze \\
  https://github.com/peterschulze04 \\
  \texttt{Peter.Schulze@student} \\\And
  Fabian Stiewe \\
  https://github.com/Fabianstw \\
  \texttt{Fabian.Stiewe@student} \\
}

%\author{
%  \textbf{First Author\textsuperscript{1}},
%  \textbf{Second Author\textsuperscript{1,2}},
%  \textbf{Third T. Author\textsuperscript{1}},
%  \textbf{Fourth Author\textsuperscript{1}},
%\\
%  \textbf{Fifth Author\textsuperscript{1,2}},
%  \textbf{Sixth Author\textsuperscript{1}},
%  \textbf{Seventh Author\textsuperscript{1}},
%  \textbf{Eighth Author \textsuperscript{1,2,3,4}},
%\\
%  \textbf{Ninth Author\textsuperscript{1}},
%  \textbf{Tenth Author\textsuperscript{1}},
%  \textbf{Eleventh E. Author\textsuperscript{1,2,3,4,5}},
%  \textbf{Twelfth Author\textsuperscript{1}},
%\\
%  \textbf{Thirteenth Author\textsuperscript{3}},
%  \textbf{Fourteenth F. Author\textsuperscript{2,4}},
%  \textbf{Fifteenth Author\textsuperscript{1}},
%  \textbf{Sixteenth Author\textsuperscript{1}},
%\\
%  \textbf{Seventeenth S. Author\textsuperscript{4,5}},
%  \textbf{Eighteenth Author\textsuperscript{3,4}},
%  \textbf{Nineteenth N. Author\textsuperscript{2,5}},
%  \textbf{Twentieth Author\textsuperscript{1}}
%\\
%\\
%  \textsuperscript{1}Affiliation 1,
%  \textsuperscript{2}Affiliation 2,
%  \textsuperscript{3}Affiliation 3,
%  \textsuperscript{4}Affiliation 4,
%  \textsuperscript{5}Affiliation 5
%\\
%  \small{
%    \textbf{Correspondence:} \href{mailto:email@domain}{email@domain}
%  }
%}

\begin{document}
\maketitle
\begin{abstract}
Large language models (LLMs) answer questions using knowledge that may change over time; however, most evaluations remain static. This work explores the demands of question recency, specifically the frequency with which answers must be updated to maintain accuracy, and the precision with which models can predict the necessary update frequency. 
We construct a \textit{generation-and-labeling} pipeline that produces recency-aware questions along two axes: stationary vs.versus non-stationary temporal behaviour, and single-event vs. multi-event dependencies. The multi-event category captures questions whose answers rely on multiple temporally connected events, including causal and purely co-occurring pairs. The single-event category focuses on questions about a single evolving process. Stationarity labels distinguish predictable update cycles from volatile changes.
Employing a Llama model, the pipeline generates and labels a $\sim$\num{1400}-question dataset, assigning recency classes, stationarity labels, and multi-event markers. Subsequently, it applies perturbations to construct more challenging temporal variants. The resulting dataset extends prior work, providing new and more demanding questions with which to evaluate LLMs on when those answers change.
\end{abstract}

\section{Introduction}
In the last couple of years, large language models (LLMs) have demonstrated remarkable capabilities across a wide range of natural language processing tasks. In particular, their performance on question answering (QA) tasks has been studied extensively \cite{fischer2024questionlargelanguagemodels,brown2020languagemodelsfewshotlearners}, with very strong results. However, a persistent challenge for QA systems is handling temporal dynamics, specifically answering questions whose correct answers change over time.

Much of this prior work largely ignores the fact that many real-world questions require information that is time-sensitive and may change frequently, such as:
\begin{itemize}
  \setlength{\itemsep}{1pt}
  \setlength{\parskip}{1pt}
  \setlength{\parsep}{1pt}
  \setlength{\topsep}{1pt}
  \item \textit{\enquote{What is the current population of Germany?}}
  \item \textit{\enquote{Who is the current president of the United States?}}
\end{itemize}
As facts evolve, answers that were once correct inevitably degrade, and outdated responses can harm user trust. Despite this, many QA evaluations remain static, implicitly assuming that factual correctness is invariant over time.

\begin{table*}[!t]
  \centering
  \label{tab:related-datasets}
  \resizebox{\textwidth}{!}{%
    \begin{tabular}{@{} l l l c c c r @{} }
      \toprule
      Dataset & Creation & KC & Multi-hop & Recency-Label & Multi-Event & \# Ques. \\
      \midrule
      TimeQA~\cite{chen2021datasetansweringtimesensitivequestions} & Templ.-Wikidata & Wikipedia & \xmark & \xmark & \xmark & \num{20000} \\
      SituatedQA~\cite{zhang2021situatedqaincorporatingextralinguisticcontexts} & Man.-Filt. & Wikipedia & \xmark & \xmark & \xmark & \num{12000} \\
      TempLama~\cite{10.1162/tacl_a_00459} & Templ./Cloze & Custom-News & \xmark & \xmark & \xmark & \num{50000} \\
      StreamingQA~\cite{liška2022streamingqabenchmarkadaptationnew} & Man.+Gen. & WMT News & \cmark & \xmark & \xmark & \num{410000} \\
      ArchivalQA~\cite{wang2022archivalqalargescalebenchmarkdataset} & Gen. & NYT Articles & \xmark & \xmark & \xmark & \num{532000} \\
      ChroniclingAmericaQA~\cite{Piryani_2024} & Gen. & Chronicling America Newspapers & \xmark & \xmark & \xmark & \num{485000} \\
      RealTimeQA~\cite{kasai2024realtimeqawhatsanswer} & News websites & News Articles & \cmark & \xmark & \xmark & $\sim$\num{5000} \\
      PATQA~\cite{meem-etal-2024-pat} & Templ.-wikidata & Wikipedia & \cmark & \xmark & \xmark & \num{6172} \\
      FreshQA~\cite{vu-etal-2024-freshllms} & Man. & Google Search & \cmark & \xmark & \xmark & \num{600} \\
      RecencyQA (\textit{unpublished}) & Man.-Filt.+Gen & Wikipedia/Wikidata & \cmark & \cmark & \xmark & \num{6115} \\
      \midrule
      \textbf{RecencyQA-Multi (\textit{ours})} & Man.-Filt.+Gen & RecencyQA+LLM & \xmark & \cmark & \cmark & \num{1432} \\
      \bottomrule
    \end{tabular}%
  }
  \caption{Overview of question answering datasets. Abbreviations: \textit{Man.}=created manually, \textit{Gen.}=Au- tomatically generated, \textit{Man.-Filt.}=filtered from other datasets, \textit{Man.+Gen.}=created by crowdsourcing and LLM generation \textit{Templ.}=created using templates, \textit{Man.-Filt.+Gen}=filtered from other datasets and LLM generation, \textit{KC}=Knowledge Corpus.}
\end{table*}

Prior work on temporal QA and time-aware modeling has highlighted this limitation from several perspectives. These include datasets for time-sensitive questions \citep{chen2021datasetansweringtimesensitivequestions}, temporal knowledge bases for language models \citep{10.1162/tacl_a_00459}, real-time QA benchmarks \citep{kasai2024realtimeqawhatsanswer}, search-augmented approaches for refreshing model knowledge \citep{vu-etal-2024-freshllms}, and self-updating present-anchored QA suites \citep{meem-etal-2024-pat}. While these efforts underscore the importance of temporal awareness, they typically treat time sensitivity as a coarse or binary property and do not explicitly characterize how frequently answers change or how complex their temporal dependencies are.

We refer to this requirement as the \textit{recency demand} of a question: the degree to which an answer must be updated over time to remain accurate. Recency demand varies across questions, depending on factors such as the predictability of change and whether an answer depends on a single evolving event or on multiple temporally related events.

A very recent paper, \emph{RecencyQA} (unpublished), introduces a dataset explicitly labeled with recency demands. The authors annotate questions according to how frequently their answers are expected to change and use this dataset to evaluate the ability of various LLMs to predict a question's recency demand. Their results indicate that, while models capture coarse distinctions between static and rapidly changing facts, accurately estimating finer-grained recency requirements remains challenging.

Building on this line of work, our approach extends prior recency-aware QA datasets along several important dimensions. Rather than relying on a fixed set of annotated questions, we introduce a generation-and-labeling pipeline that systematically expands a small seed set into a substantially richer collection of temporal questions. Starting from \num{75} input questions, the pipeline produces approximately \num{1500} recency-aware questions by varying both temporal behavior and event structure. In addition to distinguishing stationary from non-stationary temporal dynamics, we explicitly control event dependency—an aspect that has received little explicit attention in prior work—by generating single-event questions as well as multi-event questions involving two or three temporally connected events, with both causal and non-causal relationships. Each generated question is further annotated with a recency label and contextual condition under which the label applies. This structured expansion enables finer-grained analysis of temporal complexity in QA than prior datasets, which typically focus on isolated questions or coarse recency distinctions.

We start explaining the generation pipeline in \autoref{sec:dataset-creation}, followed by creating another pipeline for testing LLMs in \autoref{sec:testing-pipeline}. Using this second pipeline we then test various LLMs on the created dataset in \autoref{sec:results}. Using the results, we fine-tune a smaller model and compare it to the larger LLMs in \autoref{sec:fine_tuning}. Finally, we conclude with a discussion of findings and future work in \autoref{sec:conclusion}.

\section{Dataset Creation}
\label{sec:dataset-creation}
\begin{figure*}[!t]
    \centering
    \includegraphics[width=0.7\textwidth]{res/PipelineWorkflow.png}
    \caption[Dataset generation pipeline]{(1) Sampling \num{75} questions from the RecencyQA dataset. (2) Generating new questions by varying stationarity, number of events, and inter-event relationships. (3) Labeling each question with recency class(es) and corresponding contextual condition(s). (4) Verification of generated questions and labels by \enquote{hand} and an LLM. (5) Final dataset with \num{1411} with improvements as shown in \protect Table~\ref{tab:recency-classes} and mentioned in \protect Section~\ref{sec:question-taxonomy}.}
    \label{fig:pipeline}
\end{figure*}
Our dataset creation is closely aligned with the \textit{RecencyQA} paper's methodology, but we extend their approach by incorporating multi-event questions and refining the stationarity aspect. As illustrated in Figure~\ref{fig:pipeline}, the dataset generation pipeline comprises four main stages: question sampling, question generation, recency labeling, and verification conducted by \enquote{hand} and an LLM. This pipeline constructs (as described in the following sections) a diverse set of recency-aware questions, each annotated with appropriate recency labels and contextual conditions. For this generation and labeling stage, we used the \textit{Llama-3.3-70B-Instruct-Turbo} large language model~\cite{touvron2023llamaopenefficientfoundation}, executed via the Together AI\footnote{\href{https://www.together.ai/}{Together AI | The AI Native Cloud}} inference platform. The code and dataset are publicly available at GitHub\footnote{\href{https://github.com/Fabianstw/RecencyQA_MultiEvent}{GitHub | RecencyQA\_MultiEvent}} for reproducibility and further research.

\subsection{Question Taxonomy}
\label{sec:question-taxonomy}
As a core component of the dataset generation pipeline, we organize questions along a small set of structural dimensions, which are stored as explicit metadata for every record.
\begin{table*}[!t]
    \centering
    \resizebox{\textwidth}{!}{
        \begin{tabular}{lcl}
        \textbf{Recency Class} & \textbf{Expected time until answer change} & \textbf{Example Question} \\
        \midrule
        An-Hour & Within an hour & What is the current stock price of Apple? \\
        A-Few-Hours & Within a few hours & What is the current traffic situation on the A9 highway? \\
        A-Day & Within a day & What is todays weather forecast for Munich? \\
        A-Few-Days & Within a few days & What movies are currently trending on Netflix this week? \\
        A-Week & Within a week & What are the top-ranked songs on the Billboard chart this week? \\
        A-Few-Weeks & Within a few weeks & What is the current FIFA world ranking of the German national team? \\
        A-Month & Within a month & What is the current unemployment rate in Italy? \\
        A-Few-Months & Within a few months & What is the current inflation rate in the Eurozone? \\
        A-Year & Within a year & What is the current version of the Java programming language? \\
        A-Few-Years & Within a few years & Who is the president of the United States? \\
        Many-Years & After many years & What is the population of Germany? \\
        Never & Never changes & What is the chemical symbol for gold? \\
        \end{tabular}
    }
    \caption{The recency classes proposed by the \textit{RecencyQA} paper, along with example questions for each class.}
    \label{tab:recency-classes}
\end{table*}
\begin{itemize}
    \item \textbf{Stationarity}: Whether a questions \emph{recency requirement} is context-invariant. \emph{Stationary} questions keep the same recency label across contexts, whereas \emph{non-stationary} questions change their label when contextual conditions change.
    \item \textbf{Multi-Event}: Questions may depend on a single situation or combine information from multiple distinct or connected events. Multi-event questions increase temporal and structural complexity by requiring the integration of information across several concurrent or related processes.
    \item \textbf{Inter-event relationship}: When multiple events are involved, their relationship can be causal—where one event influences another—or purely temporal, where events co-occur in time without an assumed dependency.
    \item \textbf{Recency classes}: Each question is associated with a recency class indicating the expected timescale on which its answer may change (e.g., hours, days, or years). We adopt the recency classes proposed by the \textit{RecencyQA} paper (see Table~\ref{tab:recency-classes}).
    \item \textbf{Contextual conditions}: Recency is not absolute but depends on the assumed state of the world. Contextual conditions describe the situation under which a given recency class applies, allowing the same question (under Non-Stationary) to have different recency labels.
\end{itemize}

\subsection{Question Selection}
Having defined the question taxonomy, the next step in the pipeline is to select an initial set of seed questions from which new items are generated. These seeds serve as prompts that anchor topic, style, and temporal structure while still allowing the language model to introduce new events, domains, and cross-event interactions.

In our setting, we randomly select \num{75} questions from the original \textit{RecencyQA} dataset as seed inputs. The selected questions are evenly distributed across stationarity and recency classes. All questions are provided to the model in a structured \texttt{JSON} format (see Appendix~\ref{app:sec:input-json}).

\subsection{Question generation}
Given the set of \num{75} seed questions described in the previous section, the next stage of the pipeline expands each seed into a diverse collection of recency-aware variants that systematically explore the taxonomy introduced in Section~\ref{sec:question-taxonomy}. The goal is not to paraphrase the original questions, but to create new, semantically distinct questions that preserve similar structure while varying event composition and temporal behavior. As mentioned earlier, we utilize the \textit{Llama-3.3-70B-Instruct-Turbo}~\cite{touvron2023llamaopenefficientfoundation} model for this generation task.

\subsubsection{Generation Setup}
For each seed question, we invoke a family of prompt templates that condition the model on specific combinations of stationarity and event structure. These templates guide the model to generate questions that are either \emph{stationary} or \emph{non-stationary}, and that depend on either a \emph{single event} or on \emph{multiple} related events. For multi-event questions, we further distinguish between \emph{causal} relationships—where one event influences another—and \emph{temporal-only} relationships, where events merely co-occur in time without direct dependency. To increase task difficulty and better probe temporal reasoning capabilities (see Section~\ref{sec:results}), we additionally control the structural complexity by generating multi-event questions involving either two or three distinct events, requiring models to integrate information across multiple evolving processes.

\subsubsection{Prompt Design}
Each prompt template is designed to generate two questions per call and explicitly forbids placeholders or paraphrases of the seed, encouraging the generation of genuinely new content rather than surface-level reformulations (Appendix~\ref{app:gen-prompts}). In practice, however, the language model may very rarely return fewer than two valid questions or fail to produce a usable output. In such cases, only the successfully generated questions are retained.

\begin{table}[!t]
    \centering
    \begin{tabular}{cccc}
        \toprule
        \textbf{St.} & \textbf{\#Ev.} & \textbf{Rel.} & \textbf{Prompt} \\
        \midrule
        S  & 1 & -- & \ref{prompt:stat-single} \\
        S  & 2 & C  & \ref{prompt:stat-2-causal} \\
        S  & 2 & T  & \ref{prompt:stat-2-temporal} \\
        S  & 3 & C  & \ref{prompt:stat-3-causal} \\
        S  & 3 & T  & \ref{prompt:stat-3-temporal} \\
        \midrule
        NS & 1 & -- & \ref{prompt:nonstat-single} \\
        NS & 2 & C  & \ref{prompt:nonstat-2-causal} \\
        NS & 2 & T  & \ref{prompt:nonstat-2-temporal} \\
        NS & 3 & C  & \ref{prompt:nonstat-3-causal} \\
        NS & 3 & T  & \ref{prompt:nonstat-3-temporal} \\
        \bottomrule
    \end{tabular}
    \caption{Prompt families used for question generation. 
        \textbf{St.} = Stationarity (S = Stationary, NS = Non-stationary), 
        \textbf{\#Ev.} = number of events, 
        \textbf{Rel.} = inter-event relation (C = causal, T = temporal-only).
    }
    \label{tab:prompt-families}
\end{table}
Table~\ref{tab:prompt-families} summarizes the ten prompt families used to cover the full space of temporal structures defined by our taxonomy. Each seed question is processed by all ten prompt variants, resulting in up to twenty new questions per seed (two per prompt). Thus our complete generation process yields up to \num{1500} new questions, subject to filtering for malformed outputs as described below.

We provide, for each prompt template, representative example questions together with their accompanying contexts and the assigned recency label(s); see Appendix~\ref{app:example-questions-each-variation}.

\subsubsection{Annotation and Metadata}
\label{sec:annotation-and-metadata}
Every generated question is automatically associated with an identifier and annotated with metadata describing its stationarity, event dependency, number of events, and—when applicable—the prompted inter-event relation (causal vs. temporal-only). This process transforms each original question into a family of temporally enriched questions, yielding a dataset that supports fine-grained analysis of how temporal structure, multi-event dependency, and recency demand interact in LLM-based question answering.

\subsection{Recency and context labeling}
To apply the recency-class dimension introduced in Section~\ref{sec:question-taxonomy}, we annotate each question with one recency class (or two recency classes if non-stationary) that specify on which time scale its correct answer is expected to change. We use the twelve classes proposed by the \textit{RecencyQA} paper (Table~\ref{tab:recency-classes}) and store them as normalized strings (e.g., \texttt{An-Hour}, \texttt{A-Few-Days}).

Stationary questions receive a single label, while non-stationary questions receive two labels to capture alternative temporal regimes under different world states. For each label, we additionally elicit a one-sentence contextual condition (event, phase, or condition) that makes the question relevant under that regime. The labeling prompt prohibits reasoning and explicit timestamps and is reproduced in Appendix~\ref{app:label-prompts} via the stationary template (Appendix~\ref{prompt:label-stat}) and the non-stationary template (Appendix~\ref{prompt:label-nonstat}). Labels and conditions are merged into a structured list so downstream models can directly pair each recency class with its motivating scenario. We retain only entries whose labels parse as \texttt{JSON} and whose number of conditions matches the expected label count.

The output is stored in a structured \texttt{JSON} format (see Appendix~\ref{app:sec:output-json}) that includes the questions, recency labels, contextual conditions, and the metadata described (Section~\ref{sec:annotation-and-metadata}). In Appendix~\ref{app:example-questions-json}, we provide a concrete example of a question with its associated labels and contexts in \texttt{JSON} format.

\subsection{Verification and quality control}
To reduce noise from imperfect generations and labels, we apply a two-stage verification procedure combining an LLM-based consistency check with targeted manual review.

First, we used \textit{GPT-5.1 Codex Max} via GitHub Copilot Premium to screen all records. To fit the model's context window, we split the dataset into seven parts and asked the model to validate (i) whether the recency label(s) match the question under the provided contextual condition(s) and (ii) whether the question is sensible and grammatically well-formed, see Appendix~\ref{sec:verification-prompt} for the exact prompt. The model outputs a list of question IDs flagged as either \enquote{correct} or \enquote{incorrect}.

All items flagged as \enquote{incorrect} were then double-checked by hand and removed.
Finally, we randomly sampled \num{75} remaining questions and audited them manually; only rare edits were necessary (e.g., adjusting a label or removing an ambiguous question).

\subsection{Dataset Statistics}
We conclude this section by summarizing the resulting dataset after generation, labeling, and verification. Table~\ref{tab:dataset-stats} provides an overview of the key corpus characteristics that we use throughout our experiments and analysis. Figure~\ref{fig:recency-label-distribution} visualizes the distribution of recency labels across the entire dataset.

% Total questions (all splits) & 1411 \\
% Unique seed questions & 1373 \\
% Stationary questions & 725 \\
% Non-stationary questions & 686 \\
% Single-event questions & 286 \\
% Two-event questions & 555 \\
% Three-event questions & 570 \\
% Multi-event (causal) & 565 \\
% Multi-event (temporal-only) & 560 \\
% 1-label items & 725 \\
% 2-label items & 686 \\
% Avg. question length (words) & 22.2 \\
% Avg. context 1 length (words) & 11.2 \\
% Avg. context 2 length (words) & 12.6 \\
% recency_label1 count & 1400 \\
% recency_label2 count & 697 \\
% Most frequent labels (top-3) & [('A-Year', 373), ('A-Day', 352), ('A-Few-Days', 320)] \\
\begin{table}[!t]
    \centering
    \small
    \renewcommand{\arraystretch}{1.2}
    \setlength{\tabcolsep}{4pt}
    \begin{tabular}{l@{\hspace{0.8em}}r}
        \toprule
        \textbf{Metric} & \textbf{Value} \\
        \midrule
        Total questions (all splits) & \num{1411} \\
        Unique seed questions & \num{1373} \\
        Total Recency Classes & \num{12} \\
        \midrule
        Stationary questions & \num{725} \\
        Non-stationary questions & \num{686} \\
        \midrule
        Single-event questions & \num{286} \\
        Two-event questions & \num{555} \\
        Three-event questions & \num{570} \\
        Multi-event (causal) & \num{565} \\
        Multi-event (temporal-only) & \num{560} \\
        \midrule
        Avg. question length (tokens) & \num{22.2} \\
        Avg. context 1 length (tokens) & \num{11.2} \\
        Avg. context 2 length (tokens) & \num{12.6} \\
        \midrule
        Total recency labels & \num{2097} \\
        Most frequent labels (top-3) & \parbox[t]{0.35\linewidth}{\raggedright
            \texttt{A-Year} (\num{373}),\\
            \texttt{A-Day} (\num{352}),\\
            \texttt{A-Few-Days} (\num{320})
        } \\
        \bottomrule
    \end{tabular}
    \caption{Dataset overview statistics.}
    \label{tab:dataset-stats}
\end{table}

\begin{figure}[t]
    \centering
    \includegraphics[width=0.48\textwidth]{res/recency_labels_treemap.pdf}
    \caption{Distribution of recency labels in the final dataset.}
    \label{fig:recency-label-distribution}
\end{figure}

\section{Testing Pipeline}
\label{sec:testing-pipeline}
In this section we describe our testing pipeline for evaluating the performance of language models on our dataset, or on custom datasets. Thus, our pipeline is designed to be usable for any dataset that follows this structure and any LLM that can be accessed via \emph{Together AI}\footnote{One could also use other platforms or options than Together AI, but this would need a small refactoring of how to access and execute those models}. The pipeline consumes the unmodified \texttt{JSON} produced by the generation process (see Appendix~\ref{app:sec:output-json}).

After loading the dataset, we iterate over every question and unpack its list of temporal contexts and gold labels as provided by the schema in Appendix~\ref{app:sec:output-json}. Each entry becomes an independent evaluation instance consisting of a question, one context sentence, the corresponding recency label, and metadata describing stationarity, event dependency, and the number of events. This flattening step allows us to handle questions with one or multiple applicable contexts uniformly, while retaining the ability to later aggregate results by stationary vs.
non-stationary behavior or by the structural class of the question.

For every instance we assemble the testing prompt shown in Appendix~\ref{app:testing-prompts} and submit it to the selected model via Together AI. The prompt enforces single-label answers from the same discrete label set that was used during generation, ensuring direct comparability between model predictions and the human-authored ground truth. The temperature is fixed at \num{0,0} to minimize randomness and creativity in the outputs.

The pipeline writes one \texttt{JSONL} file per model containing all predictions, including the original question, the used context, the gold label, and the predicted label. In addition, it produces a compact summary \texttt{JSON} that reports accuracy, a tolerant accuracy (+/- one label), the number of evaluated instances, and the count of invalid responses for each model. These metrics are computed both globally and for slices such as stationary vs. non-stationary or single- vs. multi-event questions, enabling quick inspection of where a model struggles. 


\section{Results}
\label{sec:results}
Building on the question taxonomy from Section~\ref{sec:question-taxonomy} and the metadata annotations discussed in Section~\ref{sec:annotation-and-metadata}, we evaluate \textit{moonshotai/Kimi-K2-Instruct-0905} (\num{1} Trillion parameters)~\cite{moonshotai_Kimi-K2-Instruct-0905_2025}, \textit{Qwen/Qwen2.5-72B-Instruct-Turbo} (\num{72} Billion parameters)~\cite{qwen2.5,qwen2} and \textit{deepseek-ai/DeepSeek-V3} (\num{671} Billion parameters)~\cite{deepseekai2025deepseekv3technicalreport} with the testing pipeline from Section~\ref{sec:testing-pipeline}. 

\begin{table*}[t]
	\centering
	\small
	\setlength{\tabcolsep}{6pt}
	\begin{tabular}{l|cc|cc|cc|ccc|cc}
		\toprule
		& \multicolumn{2}{c|}{Overall} & \multicolumn{2}{c|}{St.} & \multicolumn{2}{c|}{Non-St.} & \multicolumn{3}{c|}{\# Events} & \multicolumn{2}{c}{Multi-event} \\
		Model & Acc & Tol. & Acc & Tol. & Acc & Tol. & 1 Acc & 2 Acc & 3 Acc & C Acc & T Acc \\
		\midrule
		Kimi & \num{33.0} & \num{62.2} & \num{42.1} & \num{66.3} & \num{28.3} & \num{60.0} & \num{35.0} & \num{31.1} & \num{33.6} & \num{34.9} & \num{29.8} \\
		Qwen & \num{45.6} & \num{77.4} & \num{62.8} & \num{81.8} & \num{36.7} & \num{75.1} & \num{45.4} & \num{44.6} & \num{46.6} & \num{47.8} & \num{43.5} \\
		Deepseek & \num{40.8} & \num{71.9} & \num{52.2} & \num{73.2} & \num{34.9} & \num{71.1} & \num{44.9} & \num{38.1} & \num{41.1} & \num{43.9} & \num{35.6} \\
		\midrule
		Total & \num{39.8} & \num{70.5} & \num{52.4} & \num{73.8} & \num{33.3} & \num{68.7} & \num{41.7} & \num{37.9} & \num{40.5} & \num{42.2} & \num{36.3} \\
		\bottomrule
	\end{tabular}
	\caption{Accuracy (Acc) and tolerant F1 accuracy (Tol., $\pm 1$ label; computed as the tolerant F1 metric). St.: stationary; Non-St.: non-stationary; C: causal; T: temporal-only.}
	\label{tab:model-overview}
\end{table*}

\subsection{Overall Performance}
Qwen2.5-72B attains the strongest accuracy (\num{45.6}\unit{\%}) and tolerant F1 accuracy (\num{77.4}\unit{\%}). DeepSeek-V3 trails by roughly five absolute points, while the Kimi-K2 instruction model stays below \num{35}\unit{\%} accuracy despite a comparable tolerant score. Invalid generations remain negligible, confirming that the constrained testing prompt in Appendix~\ref{app:testing-prompts} keeps responses well-formed.

\subsection{Impact of Stationarity}
Stationarity is the clearest driver of variance. Each model loses between \num{13} and \num{26} percentage points when moving from stationary to non-stationary questions, mirroring the temporal volatility highlighted during annotation (Section~\ref{sec:annotation-and-metadata}). Qwen drops from \num{62.8}\% to \num{36.7}\% accuracy, indicating that even large instruction models struggle when the required recency hinges on punctual events rather than cyclical updates. Kimi suffers the steepest relative decline ($-13.8$ points) because its stationary accuracy is already modest, whereas DeepSeek maintains low-\num{30}s performance on non-stationary prompts despite exceeding \num{52}\% on the stable slice. These gaps confirm that the dataset captures the adaptive reasoning behaviour targeted in Section~\ref{sec:question-taxonomy}.

\subsection{Mutli-Event Question Performance}
In the final four columns of Table~\ref{tab:model-overview}, we break down model performance on multi-event questions by number of events and generation type: causal (C) vs. temporal-only (T). For all three models we don't observe significant performance differences based on the number of events involved in the question. The tolerant accuracy has a slightly more downward trend as the number of events increases (from \num{80.3}\unit{\%} to \num{66.2}\unit{\%} (two events) and \num{69.3}\unit{\%} (three events) (aggregated)). This does indicate that multi-event questions are more difficult than single-event questions, but the number of events itself does not seem to have a strong influence. 

When comparing causal vs. temporal-only multi-event questions, we observe that all models perform better on causal questions. Qwen achieves \num{47.8}\unit{\%} accuracy on causal questions compared to \num{43.5}\unit{\%} on temporal-only questions. Kimi and DeepSeek show similar trends with \num{34.9}\unit{\%} vs. \num{29.8}\unit{\%} and \num{43.9}\unit{\%} vs. \num{35.6}\unit{\%}, respectively. This suggests that models may find it easier to reason about cause-and-effect relationships when determining recency, as opposed to purely temporal relationships. Though the differences are not very large, they are consistent across all models.


\begin{figure}
  \centering
  \includegraphics[width=\linewidth]{res/aggregated_confusion_heatmap.pdf}
  \caption{Model-averaged confusion matrix over all 12 recency labels, showing recall per label.}
  \label{fig:recency-confusion}
\end{figure}
\subsection{Label-wise Behaviour and Error Direction}
Per-label statistics expose systematic blind spots. All three systems excel at the shortest horizons (e.g., Qwen reaches recall \num{55}\unit{\%} on \texttt{A-Few-Hours} and Kimi exceeds \num{73}\unit{\%}), yet accuracy collapses for intermediate windows: \texttt{A-Week} recall never surpasses \num{10}\unit{\%}, and \texttt{A-Month} remains below \num{20}\unit{\%} for Kimi and DeepSeek. Long-horizon labels such as \texttt{Many-Years} and \texttt{Never} also exhibit low precision (Qwen's best F$_1$ for \texttt{Never} is \num{6.5}\unit{\%}). A condensed, model-averaged confusion heatmap makes these failure modes visible; we therefore refer to Figure~\ref{fig:recency-confusion} for the aggregated confusion matrix over all 12 labels.

\begin{figure}
  \centering
  \includegraphics[width=\linewidth]{res/error_direction_barplot.pdf}
  \caption{Directional error counts per model, contrasting \enquote{too early} vs. \enquote{too late} predictions.}
  \label{fig:error-direction}
\end{figure}

Directional errors emphasise another imbalance. Kimi produces \num{801} \enquote{too early} predictions versus \num{591} \enquote{too late}, Qwen \num{695} vs. \num{439}, and DeepSeek \num{652} vs. \num{582}. The models thus prefer overly fresh information, underestimating how slowly certain questions evolve. This bias matters for downstream systems that rely on recency signals to schedule refreshes: adopting a \enquote{check too often} policy could waste computation, whereas missing late updates risks outdated answers. To visualize this effect we refer to Figure~\ref{fig:error-direction}, a bar chart contrasting the error directions per model.

Finally, tolerant accuracy being between \num{20} and \num{30} points higher than accuracy across all models shows that most mistakes deviate by only one label. While encouraging, this plateau also signals that the discrete label borders defined in Section~\ref{sec:question-taxonomy} remain hard to recover without explicit temporal reasoning, motivating future work on richer reasoning prompts or retrieval-augmented pipelines.


\section{LLM Fine-Tuning}
\label{sec:fine_tuning}
\begin{frame}
  \frametitle{Fine-tuning setup}
  \begin{columns}
    \begin{column}{0.37\textwidth}
      \begin{itemize}
      \item Smaller Qwen model (\num{14}B)
      \item Split Dataset \num{70}/\num{15}/\num{15} 
      \item Finetuned via Together AI
      \item Compared to previous models
    \end{itemize}
    \end{column}
    \begin{column}{0.63\textwidth}
      \vspace{-2em}
      \begin{figure}[h]
        \centering
        \includegraphics[width=\linewidth]{_images/FineTuneOverview.png}
      \end{figure}
    \end{column}
  \end{columns}
\end{frame}

\begin{frame}[c]
  \frametitle{Fine-tuning results (reduced test set)}
  \scriptsize
  \centering
  \resizebox{\textwidth}{!}{%
  \begin{tabular}{l|cc|cc|cc|cc|cc}
    \toprule
    & \multicolumn{2}{c|}{Overall} & \multicolumn{2}{c|}{St.} & \multicolumn{2}{c|}{Non-St.} & \multicolumn{2}{c|}{Single-event} & \multicolumn{2}{c}{Multi-event} \\
    Model & Acc & Tol. & Acc & Tol. & Acc & Tol. & Acc & Tol. & Acc & Tol. \\
    \midrule
    Qwen2.5-14B (FT) & \num{40.9} & \num{69.6} & \num{55.1} & \num{72.9} & \num{33.5} & \num{68.0} & \num{41.5} & \num{78.5} & \num{40.7} & \num{67.3} \\
    Kimi & \num{35.5} & \num{64.9} & \num{46.3} & \num{70.4} & \num{29.8} & \num{62.0} & \num{29.7} & \num{65.6} & \num{36.9} & \num{64.7} \\
    Qwen2.5-72B & \num{46.5} & \num{76.1} & \num{63.9} & \num{82.4} & \num{37.4} & \num{72.8} & \num{46.2} & \num{87.7} & \num{46.6} & \num{73.1} \\
    DeepSeek-V3 & \num{41.4} & \num{69.7} & \num{50.9} & \num{71.3} & \num{36.4} & \num{68.9} & \num{38.5} & \num{70.8} & \num{42.2} & \num{69.5} \\
    \bottomrule
  \end{tabular}%
  }
\end{frame}

% \begin{frame}[c]
%   \frametitle{Fine-tuning results — accuracy barplot}
%   \centering
%   \includegraphics[width=0.85\linewidth]{../Paper/res/finetuned_accuracy_barplot.pdf}
% \end{frame}

\begin{frame}[plain]
  \frametitle{Parameter efficiency}
  \centering
  \includegraphics[width=0.85\linewidth]{../Paper/res/finetuned_param_normalized.pdf}
\end{frame}


\section{Summary and Outlook}
\label{sec:conclusion}
Having detailed the dataset construction, evaluation, and fine-tuning analysis, we now synthesize the main takeaways.

\subsection{Conclusion}
We introduced \textit{RecencyQA-Multi}, a systematically generated extension of the original RecencyQA corpus that explicitly varies stationarity, the number of interdependent events, and the relationships between them (Section~\ref{sec:dataset-creation}). Starting from only \num{75} seed questions, our controlled prompting and verification pipeline yielded \num{1411} questions with fine-grained recency labels and contextual conditions, providing a richer testbed for studying temporal reasoning.

To gauge how well current models exploit this structure, we designed a reusable evaluation pipeline (Section~\ref{sec:testing-pipeline}) and benchmarked state-of-the-art instruction models (Section~\ref{sec:results}). Despite tolerant accuracies exceeding \num{70}\unit{\%}, all models misclassify most instances under the strict metric, and accuracy drops sharply for non-stationary and multi-event questions. Qwen2.5-72B performs best overall, yet still struggles to distinguish medium-horizon labels and consistently predicts overly fresh answers, underscoring the difficulty of calibrating temporal priors even for large LLMs.

Fine-tuning a smaller model confirms that adaptation can close much of the scale gap: the \textit{Qwen2.5-14B-Instruct-recency} model reaches \num{40.9}\unit{\%} accuracy and \num{69.6}\unit{\%} tolerant F$_1$ on the reduced test set, matching DeepSeek-V3 and surpassing Kimi-K2 while trailing Qwen2.5-72B. When normalized by parameter count, the fine-tuned model delivers the strongest performance per parameter (Figure~\ref{fig:finetune-param-efficiency}), underscoring that targeted fine-tuning yields substantial efficiency gains.

\subsection{Future Research}
Several directions for future work could strengthen both the reliability and expressiveness of recency labels. A natural next step is to construct a dataset in which labels are assigned by humans at scale, ideally with multiple independent annotators per question. This could be facilitated through a lightweight web interface, and the final label could be derived from an aggregation rule such as the median or majority vote to reduce individual bias and noise. This would allow a more accurate evaluation of model performance against human judgment.

In addition, model ensembles offer a complementary avenue: multiple models could be run in parallel and the most consistently predicted label could be used as the final decision. In cases where the ensemble remains inconclusive, additional models (or a second-stage decision process) could be introduced, informed by the earlier model outputs to guide disambiguation. Potentially, this could yield more robust recency estimates by leveraging diverse model perspectives.

Finally, instead of treating recency prediction as a single hard-label task, it may be more appropriate to model ambiguity explicitly by using a distribution over labels as the target. For example, if annotators disagree between \enquote{A-Week} and \enquote{A-Month}, the training signal could reflect this uncertainty via proportional target probabilities. Such soft targets would be less brittle than fixed labels and could better capture the inherently fuzzy boundary between time ranges.



% Bibliography entries for the entire Anthology, followed by custom entries
\bibliography{custom}

\newpage
\appendix
\onecolumn
\section{JSON-structures}
\label{app:json-structure}
\subsection{Input JSON format for the pipeline}
\label{app:sec:input-json}
\begin{figure}[h]
\begin{lstlisting}[language=JSON]
[
  {
    "q_id": "<string | int>",
    "question": "<string>"
  }
]
\end{lstlisting}
\caption{JSON structure for the RecencyQA\_MultiEvent pipeline input.}
\end{figure}

\subsection{Output JSON format from the pipeline}
\label{app:sec:output-json}
\begin{figure}[h]
\begin{lstlisting}[language=JSON]
[
  {
    "q_id": 3,
    "question": "...",
    "event_dependency": "Single-Event | Multi-Event",
    "num_events": 1 | 2 | 3,
    "generation_type": "causal | temporal_only",   // only for Multi-Event
    "stationary": "YES | NO",
    "labels": [
      {
        "recency_label1": <Recency label>,
        "context1": "short natural-language description of the assumed situation"
      },
      {
        "recency_label2": <Recency label>,
        "context2": "short natural-language description of the assumed situation"
      }
      // second entry only appears when stationary = NO
    ]
  }
]
\end{lstlisting}
\caption{JSON structure for the RecencyQA\_MultiEvent pipeline output.}
\end{figure}

\newpage

\section{Prompt Templates}
\label{app:prompts}
\subsection{Generation prompts}
\label{app:gen-prompts}

% \subsubsection{Stationary generation}\label{app:gen-prompts-stationary}

\subsubsection{Stationary single-event}\label{prompt:stat-single}
\begin{promptblock}
You generate STATIONARY temporal questions.

Definition (internal):
- A stationary question has a stable temporal update behavior.
- The answer changes over time, but the timespan how often the answer must be updated
  would remain the same regardless of when the question is asked.

Task:
Generate EXACTLY 2 stationary temporal questions focusing on ONE event or process.

Rules:
- Must rely on real-world phenomena that change over time.
- Must be stable, cyclical, predictable, or rhythm-based.
- Must NOT paraphrase the examples.
- Must NOT use placeholders.
- Do NOT mention stationarity in the question.

Examples:
{examples}

Return JSON:
{
  "questions": ["q1","q2"]
}
\end{promptblock}

\subsubsection{Stationary multi-event (causal)}\label{prompt:stat-2-causal}
\begin{promptblock}
You generate STATIONARY multi-event temporal questions.

Task:
Generate EXACTLY 2 stationary temporal questions involving TWO events that are causally related.

Rules:
- Must involve at least TWO distinct temporal events.
- Events must have a clear cause-effect relationship.
- Temporal behavior must be stable, predictable, cyclical, or regular.
- Must rely on real-world change.
- Do NOT paraphrase examples.
- Do NOT use placeholders.
- Do NOT mention stationarity.

Examples:
{examples}

Return JSON:
{
  "questions": ["q1","q2"]
}
\end{promptblock}

\subsubsection{Stationary multi-event (temporal-only)}\label{prompt:stat-2-temporal}
\begin{promptblock}
You generate STATIONARY multi-event temporal questions.

Task:
Generate EXACTLY 2 stationary temporal questions involving TWO events that are ONLY temporally connected.

Rules:
- Events must be from clearly different real-world domains.
- Events must NOT influence each other causally.
- Must NOT belong to the same topic, organization, or event series.
- Temporal behavior must be stable, predictable, cyclical, or regular.
- Must rely on real-world change.
- Do NOT paraphrase examples.
- Do NOT use placeholders.
- Do NOT mention stationarity.

Examples:
{examples}

Return JSON:
{
  "questions": ["q1","q2"]
}
\end{promptblock}

\subsubsection{Stationary three-event (causal)}\label{prompt:stat-3-causal}
\begin{promptblock}
You generate STATIONARY multi-event temporal questions.

Task:
Generate EXACTLY 2 stationary temporal questions involving THREE events
that are causally related in a chain or network.

Rules:
- Must involve EXACTLY THREE distinct temporal events.
- Events must have clear cause-effect relationships.
- Temporal behavior must be stable, predictable, cyclical, or regular.
- Must rely on real-world change.
- Do NOT paraphrase examples.
- Do NOT use placeholders.
- Do NOT mention stationarity.

Examples:
{examples}

Return JSON:
{
  "questions": ["q1", "q2"]
}
\end{promptblock}

\subsubsection{Stationary three-event (temporal-only)}\label{prompt:stat-3-temporal}
\begin{promptblock}
You generate STATIONARY multi-event temporal questions.

Task:
Generate EXACTLY 2 stationary temporal questions involving THREE events
that are ONLY temporally connected.

Rules:
- Must involve EXACTLY THREE distinct events.
- Events must be from clearly different real-world domains.
- Events must NOT influence each other causally.
- Must NOT belong to the same topic, organization, or event series.
- Temporal behavior must be stable, predictable, cyclical, or regular.
- Must rely on real-world change.
- Do NOT paraphrase examples.
- Do NOT use placeholders.
- Do NOT mention stationarity.

Examples:
{examples}

Return JSON:
{
  "questions": ["q1", "q2"]
}
\end{promptblock}

% \subsubsection{Non-stationary generation}\label{app:gen-prompts-nonstat}

\subsubsection{Non-stationary single-event}\label{prompt:nonstat-single}
\begin{promptblock}
You generate NON-STATIONARY temporal questions.

Definition (internal):
- A non-stationary question has unstable temporal update behavior.
- How frequently the answer must be updated depends strongly on WHEN the question is asked.
- OR the question is only relevant within short, event-dependent windows.

Task:
Generate EXACTLY 2 non-stationary temporal questions focusing on ONE event.

Rules:
- Must rely on quickly evolving or unstable processes.
- Must be meaningful and real-world.
- Do NOT paraphrase examples.
- Do NOT use placeholders.
- Do NOT mention non-stationarity explicitly.

Examples:
{examples}

Return JSON:
{
  "questions": ["q1","q2"]
}
\end{promptblock}

\subsubsection{Non-stationary multi-event (causal)}\label{prompt:nonstat-2-causal}
\begin{promptblock}
You generate NON-STATIONARY multi-event temporal questions.

Task:
Generate EXACTLY 2 non-stationary temporal questions involving TWO events that are causally related.

Rules:
- At least one event must be unstable, unpredictable, or highly time-sensitive.
- Events must have a clear cause-effect relationship.
- Must rely on real-world temporal change.
- Must involve at least TWO distinct temporal events.
- Do NOT paraphrase examples.
- Do NOT use placeholders.
- Do NOT mention non-stationarity.

Examples:
{examples}

Return JSON:
{
  "questions": ["q1","q2"]
}
\end{promptblock}

\subsubsection{Non-stationary multi-event (temporal-only)}\label{prompt:nonstat-2-temporal}
\begin{promptblock}
You generate NON-STATIONARY multi-event temporal questions.

Task:
Generate EXACTLY 2 non-stationary temporal questions involving TWO events that are ONLY temporally connected.

Rules:
- Events must be from clearly different real-world domains.
- Events must NOT influence each other causally.
- Must NOT belong to the same topic, organization, or event series.
- At least one event must be unstable, unpredictable, or highly time-sensitive.
- Must rely on real-world temporal change.
- Must involve at least TWO distinct temporal events.
- Do NOT paraphrase examples.
- Do NOT use placeholders.
- Do NOT mention non-stationarity.

Examples:
{examples}

Return JSON:
{
  "questions": ["q1","q2"]
}
\end{promptblock}

\subsubsection{Non-stationary three-event (causal)}\label{prompt:nonstat-3-causal}
\begin{promptblock}
You generate NON-STATIONARY multi-event temporal questions.

Task:
Generate EXACTLY 2 non-stationary temporal questions involving THREE events
that are causally related.

Rules:
- Must involve EXACTLY THREE distinct temporal events.
- At least one event must be unstable, unpredictable, or highly time-sensitive.
- Events must have clear cause-effect relationships.
- Must rely on real-world temporal change.
- Do NOT paraphrase examples.
- Do NOT use placeholders.
- Do NOT mention non-stationarity.

Examples:
{examples}

Return JSON:
{
  "questions": ["q1", "q2"]
}
\end{promptblock}

\subsubsection{Non-stationary three-event (temporal-only)}\label{prompt:nonstat-3-temporal}
\begin{promptblock}
You generate NON-STATIONARY multi-event temporal questions.

Task:
Generate EXACTLY 2 non-stationary temporal questions involving THREE events
that are ONLY temporally connected.

Rules:
- Must involve EXACTLY THREE distinct temporal events.
- Events must be from clearly different real-world domains.
- Must NOT influence each other causally.
- At least one event must be unstable, unpredictable, or highly time-sensitive.
- Must rely on real-world temporal change.
- Do NOT paraphrase examples.
- Do NOT use placeholders.
- Do NOT mention non-stationarity.

Examples:
{examples}

Return JSON:
{
  "questions": ["q1", "q2"]
}
\end{promptblock}

\subsection{Verification Prompt}
\label{sec:verification-prompt}
\begin{promptblock}
Can you please check for every single question in this dataset, if:
- the recency_label1 (and if available recency_label2) is correct for the question and context1 (if available context2)
- and if the questions makes sense in general (gramnatically)
Write the id into correct if correct, otherwise into incorrect.
\end{promptblock}

\subsection{Labeling prompts}
\label{app:label-prompts}

\subsubsection{Stationary labeling}\label{prompt:label-stat}
\begin{promptblock}
Analyze this temporal question and produce temporal labels:

"{question}"

Your tasks (internal reasoning only, output JSON only):

1. Provide ONE recency label:
{
  "An-Hour": "changes within one hour",
  "A-Few-Hours": "changes within a few hours",
  "A-Day": "changes within one day",
  "A-Few-Days": "changes within a few days",
  "A-Week": "changes within one week",
  "A-Few-Weeks": "changes within a few weeks",
  "A-Month": "changes within one month",
  "A-Few-Months": "changes within a few months",
  "A-Year": "changes within one year",
  "A-Few-Years": "changes within a few years",
  "Many-Years": "changes after 10 or more years",
  "Never": "never changes"
}

2. Based on the selected label, write a short temporal context (ONE sentence) describing the current event, phase, or condition in which the question is asked. This does not have to be related to the question causally.
   - Must describe an EVENT, PHASE or CONDITION, taking place when the question becomes relevant.
   - No specific years.
   - No meta reasoning.
   - No "current"
   - No "the question is asked"

Return JSON:
{
  "recency_list": ["label1"],
  "context_list": ["context1"]
}
\end{promptblock}

\subsubsection{Non-stationary labeling}\label{prompt:label-nonstat}
\begin{promptblock}
Analyze this temporal question and produce temporal labels:

"{question}"

Your tasks (internal reasoning only, output JSON only):

1. Provide TWO recency labels (for two different realistic temporal situations), choose from:
{
  "An-Hour": "changes within one hour",
  "A-Few-Hours": "changes within a few hours",
  "A-Day": "changes within one day",
  "A-Few-Days": "changes within a few days",
  "A-Week": "changes within one week",
  "A-Few-Weeks": "changes within a few weeks",
  "A-Month": "changes within one month",
  "A-Few-Months": "changes within a few months",
  "A-Year": "changes within one year",
  "A-Few-Years": "changes within a few years",
  "Many-Years": "changes after 10 or more years",
  "Never": "never changes"
}

2. For each selected label provide ONE short temporal context (ONE sentence each) describing the current event, phase, or condition in which the question is asked. This does not have to be related to the question causally.
  - Each context must describe a different EVENT, PHASE or CONDITION, taking place when the question becomes relevant.
   - No specific years.
   - No meta reasoning.
   - No "current"
   - No "the question is asked"

Return JSON:
{
  "recency_list": ["label1", "label2"],
  "context_list": ["context1", "context2"]
}
\end{promptblock}

\subsection{Testing Prompts}
\label{app:testing-prompts}
\begin{promptblock}
You are an expert in temporal reasoning.

Given the following question and its temporal context,
classify the needed recency of the data for the answer.

Question:
{question}

Context:
{context}

Choose exactly one label from:
[An-Hour, A-Few-Hours, A-Day, A-Few-Days, A-Week, A-Few-Weeks,
 A-Month, A-Few-Months, A-Year, A-Few-Years, Many-Years, Never]

Answer ONLY with the label.
DO NOT provide any explanations or additional text.
\end{promptblock}

\newpage
\section{Example Questions}
\label{app:example-questions}
\subsection{Specific Questions}
\label{app:example-questions-each-variation}

\begin{questionexample}[title=Stationary Single Event]
  \question{What is the current water level in the Amazon River during the wet season?}
  \context{Heavy rainfall is occurring in the Amazon basin during the wet season.}
  \recencylabel{A-Few-Days}
\end{questionexample}

\begin{questionexample}[title=Stationary Two Events Causal]
  \question{Do the daily tidal patterns in the Bay of Fundy trigger the opening and closing of the Annapolis Royal tidal power plant?}
  \context{Tidal patterns are being closely monitored by researchers at the Bay of Fundy.}
  \recencylabel{A-Day}
\end{questionexample}

\begin{questionexample}[title=Stationary Two Events Temporal-Only]
  \question{Are the opening hours of the Louvre Museum in Paris synchronized with the daily tidal patterns in the Bay of Fundy?}
  \context{Tourist season is in full swing in Europe.}
  \recencylabel{Never}
\end{questionexample}

\begin{questionexample}[title=Stationary Three Events Causal]
  \question{Will the rise in global temperatures cause more frequent heatwaves in Australia, which in turn increase the risk of bushfires in the region?}
  \context{A prolonged period of climate change is being observed globally.}
  \recencylabel{Many-Years}
\end{questionexample}

\begin{questionexample}[title=Stationary Three Events Temporal-Only]
  \question{What is the current time in New York when the Tokyo Stock Exchange opens and the first tennis match at Wimbledon begins?}
  \context{Financial markets and sports events are in full swing during summer mornings.}
  \recencylabel{A-Few-Hours}
\end{questionexample}

\newpage
\begin{questionexample}[title=Non-Stationary Single Event]
  \question{What is the current status of the wildfire in California?}
  \contextii{Firefighters are working to contain the blaze during intense heat waves.}{Emergency responders are assessing damage after a night of strong winds.}
  \labelsii{A-Few-Hours}{A-Day}
\end{questionexample}

\begin{questionexample}[title=Non-Stationary Two Events Causal]
  \question{How do weather forecasters predict the trajectory of a hurricane after it makes landfall, given the rapid change in atmospheric conditions?}
  \contextii{Hurricane warnings have been issued for several coastal cities as the storm approaches land.}{Emergency responders are scrambling to prepare evacuation routes as the hurricane's outer rain bands start to affect the area.}
  \labelsii{A-Few-Days}{A-Day}
\end{questionexample}

\begin{questionexample}[title=Non-Stationary Two Events Temporal-Only]
  \question{What was the status of the COVID-19 pandemic in the United States when the Perseverance rover landed on Mars?}
  \contextii{Scientists are analyzing the latest wave of COVID-19 variants.}{The world is reflecting on the pandemic's impact during a global health conference.}
  \labelsii{A-Few-Months}{A-Year}
\end{questionexample}

\begin{questionexample}[title=Non-Stationary Three Events Causal]
  \question{What impact will the unexpected power outage have on the scheduled software update and the subsequent data backup process, considering the backup window is limited to a narrow time frame?}
  \contextii{The IT team is rushing to meet a tight project deadline.}{A severe thunderstorm warning has been issued for the area.}
  \labelsii{A-Few-Hours}{A-Day}
\end{questionexample}

\begin{questionexample}[title=Non-Stationary Three Events Temporal-Onlys]
  \question{Does the timing of the annual Monaco Grand Prix overlap with the blooming of cherry blossoms in Japan and the announcement of the Nobel Prize winners?}
  \contextii{The Monaco Grand Prix is nearing its traditional date in late spring.}{Tourists are finalizing their travel itineraries for the upcoming cherry blossom festival in Japan.}
  \labelsii{A-Year}{A-Few-Days}
\end{questionexample}


\newpage
\subsection{Example Question JSON Output}
\label{app:example-questions-json}
\begin{lstlisting}[language=JSON]
  {
    "q_id": 12,
    "question": "How many people have been reported injured or missing since the last update on the hurricane landfall in Florida?",
    "event_dependency": "Single-Event",
    "num_events": 1,
    "stationary": "NO",
    "labels": [
      {
        "recency_label1": "A-Few-Hours",
        "context1": "Emergency responders are scrambling to evacuate coastal areas as the hurricane makes landfall."
      },
      {
        "recency_label2": "A-Day",
        "context2": "Relief efforts are underway as communities begin to assess the damage from the storm."
      }
    ]
  }
\end{lstlisting}

\end{document}
