%% RecencyQA-Multi presentation (main file)
\documentclass[11pt,t,usepdftitle=false,aspectratio=169]{beamer}

\usetheme[nototalframenumber,foot,logo,nosectiontitlepage]{uibk}

%% Minimal extras for tables/figures
\usepackage{booktabs}
\usepackage{graphicx}
\usepackage{pifont}
\usepackage{xstring}

% Rounded block styling
\usepackage[most]{tcolorbox}

\usepackage{tikz}
\usetikzlibrary{calc,tikzmark}
\tcbuselibrary{skins}
\usetikzlibrary{overlay-beamer-styles,positioning,arrows.meta}

\usepackage{colortbl}
\usepackage{xcolor}

% For \num
\usepackage{siunitx}
% For quotes
\usepackage{csquotes}

\usepackage{pgfpages}
% \setbeameroption{show notes on second screen=right}

\newcommand{\cmark}{\ding{51}}
\newcommand{\xmark}{\ding{55}}

%% Initialize
\newcommand{\nextsectionname}{} 

%% Update at each section
\let\oldsection\section
\renewcommand{\section}[1]{%
    \gdef\nextsectionname{#1}% store the upcoming section
    \oldsection{#1}%
}

\title[RecencyQA-Multi]{When Do Answers Change? Estimating Question Recency Demands in QA with Multi-Events}
\subtitle{Recency-aware QA dataset, pipeline, and results}
% \URL{https://github.com/Fabianstw/RecencyQA\_MultiEvent}

\author[Schulze \& Stiewe]{Peter Schulze \and Fabian Stiewe}

\footertext{RecencyQA-Multi-Event}
\date{\today}

\headerimage{6}

% Needs: \usepackage{xstring}

% Layout parameters
\newlength{\agendamarkerw}
\setlength{\agendamarkerw}{1.1em}

\newlength{\agendasep}
\setlength{\agendasep}{0.6em}

% Compact rows
\renewcommand{\arraystretch}{0.85}

\newcommand{\agendaentry}[1]{%
  \noindent
  \IfStrEq{\nextsectionname}{#1}{%
    \begin{tabular}{@{}p{\agendamarkerw}@{\hspace{\agendasep}}p{0.78\linewidth}@{}}
      {\color{uibkblue}\Large$\blacktriangleright$} &
      {\color{uibkblue}\bfseries\large #1}
    \end{tabular}%
  }{%
    \begin{tabular}{@{}p{\agendamarkerw}@{\hspace{\agendasep}}p{0.78\linewidth}@{}}
      {\color{gray!60}\small$\bullet$} &
      {\color{gray!75}\normalsize #1}
    \end{tabular}%
  }%
}

\AtBeginSection[]{
\begin{frame}[plain]
  \vfill
  \centering
  {\LARGE\color{uibkblue}\bfseries Agenda}\par
  \vspace{0.6em}

  \begin{minipage}{0.82\linewidth}
    \begin{flushleft}

      % No decorative line anymore — clean layout
      \agendaentry{Motivation and Prior Work}\\[0.25em]
      \agendaentry{Dataset Creation}\\[0.25em]
      \agendaentry{Testing Pipeline}\\[0.25em]
      \agendaentry{Results}\\[0.25em]
      \agendaentry{Fine-Tuning}\\[0.25em]
      \agendaentry{Conclusion and Outlook}

    \end{flushleft}
  \end{minipage}

  \vfill
\end{frame}
}

% -----------------------------
% Rounded `block` replacement (StackExchange-inspired)
% -----------------------------

\makeatletter

\newtcbox{\titlebox}{%
  enhanced,
  overlay={% small decorative triangle on the right of the title box
    \draw[uibkblue,fill=uibkblue](frame.south east)--+(0,.2)to[bend right]+(.2,-0)--cycle;},
  colback=uibkblue,
  top=-1pt,bottom=-2pt,left=2pt,right=2pt,
  boxrule=1pt,
  colframe=uibkblue,
  sharp corners=south,
  colupper=white,
  fontupper=\bfseries
}

\newtcolorbox{myblock}[1][]{%
  enhanced,
  left=6pt,
  right=6pt,
  colframe=uibkblue,
  boxrule=0.9pt,
  colback=uibkblue!8,
  arc=6pt,
  sharp corners=northwest,
  overlay={%
    \def\myblock@tempa{#1}%
    \ifx\myblock@tempa\@empty
    \else
      \draw[uibkblue,fill=uibkblue]($(frame.north west)+(.2pt,-.2pt)$)--+(.1,0)to[bend right]+(-0,-.1)--cycle;
      \node[anchor=south west,inner sep=0pt,outer sep=0pt] at (frame.north west){\titlebox{#1}};
    \fi
  }
}
\makeatother

% Make beamer's `block` use the new rounded style: pass title as optional arg
\let\oldblock\block
\let\endoldblock\endblock
\renewenvironment{block}[1]{\begin{myblock}[#1]}{\end{myblock}}

\begin{document}

\maketitle

\section{Motivation and Prior Work}
\begin{frame}[plain]
  \frametitle{Motivation}
  \centering
  \includegraphics[width=0.95\linewidth]{helper_scripts/volatility_reveal_5.pdf}
\end{frame}

\begin{frame}
  \frametitle{Why recency demand?}
  \begin{itemize}
    \item Answer to questions change 
    \item \textbf{Recency demand}: refresh rate; to keep answer up-to-date
    \vspace{3em}
  \end{itemize}
  \vspace{1em}
  \begin{block}{Goal}
    Build datasets that stress temporal reasoning and evaluate \textbf{LLM} performance on recency demand.
  \end{block}
\end{frame}

\begin{frame}[c]
  \frametitle{Related datasets}
  \begin{center}
    \scriptsize
    \centering
    \begin{tabular}{@{} l l l c c c r @{}}
      \toprule
      Dataset & Creation & Knowledge & KC & Recency & Multi-Event & \#Q \\
      \midrule
      TimeQA & Templ. & Wikipedia & \xmark & \xmark & \xmark & \num{20000} \\
      StreamingQA & Man.+Gen. & News & \cmark & \xmark & \xmark & \num{410000} \\
      RealTimeQA & News & News & \cmark & \xmark & \xmark & \num{5000} \\
      PATQA & Templ. & Wikipedia & \cmark & \xmark & \xmark & \num{6172} \\
      FreshQA & Manual & Web & \cmark & \xmark & \xmark & \num{600} \\
      RecencyQA (orig.) & Man.+Gen. & Wiki & \cmark & \cmark & \xmark & \num{6115} \\
      \midrule
      RecencyQA-Multi (ours) & Man.+Gen. & RecencyQA+LLM & \xmark & \cmark & \cmark & \num{1411} \\
      \bottomrule
    \end{tabular}
  \end{center}
\end{frame}


\section{Dataset Creation}
\begin{figure*}[!t]
    \centering
    \includegraphics[width=0.7\textwidth]{res/PipelineWorkflow.png}
    \caption[Dataset generation pipeline]{(1) Sampling \num{75} questions from the RecencyQA dataset. (2) Generating new questions by varying stationarity, number of events, and inter-event relationships. (3) Labeling each question with recency class(es) and corresponding contextual condition(s). (4) Verification of generated questions and labels by \enquote{hand} and an LLM. (5) Final dataset with \num{1411} with improvements as shown in \protect Table~\ref{tab:recency-classes} and mentioned in \protect Section~\ref{sec:question-taxonomy}.}
    \label{fig:pipeline}
\end{figure*}
Our dataset creation is closely aligned with the \textit{RecencyQA} paper's methodology, but we extend their approach by incorporating multi-event questions and refining the stationarity aspect. As illustrated in Figure~\ref{fig:pipeline}, the dataset generation pipeline comprises four main stages: question sampling, question generation, recency labeling, and verification conducted by \enquote{hand} and an LLM. This pipeline constructs (as described in the following sections) a diverse set of recency-aware questions, each annotated with appropriate recency labels and contextual conditions. For this generation and labeling stage, we used the \textit{Llama-3.3-70B-Instruct-Turbo} large language model~\cite{touvron2023llamaopenefficientfoundation}, executed via the Together AI\footnote{\href{https://www.together.ai/}{Together AI | The AI Native Cloud}} inference platform. The code and dataset are publicly available at GitHub\footnote{\href{https://github.com/Fabianstw/RecencyQA_MultiEvent}{GitHub | RecencyQA\_MultiEvent}} for reproducibility and further research.

\subsection{Question Taxonomy}
\label{sec:question-taxonomy}
As a core component of the dataset generation pipeline, we organize questions along a small set of structural dimensions, which are stored as explicit metadata for every record.
\begin{table*}[!t]
    \centering
    \resizebox{\textwidth}{!}{
        \begin{tabular}{lcl}
        \textbf{Recency Class} & \textbf{Expected time until answer change} & \textbf{Example Question} \\
        \midrule
        An-Hour & Within an hour & What is the current stock price of Apple? \\
        A-Few-Hours & Within a few hours & What is the current traffic situation on the A9 highway? \\
        A-Day & Within a day & What is todays weather forecast for Munich? \\
        A-Few-Days & Within a few days & What movies are currently trending on Netflix this week? \\
        A-Week & Within a week & What are the top-ranked songs on the Billboard chart this week? \\
        A-Few-Weeks & Within a few weeks & What is the current FIFA world ranking of the German national team? \\
        A-Month & Within a month & What is the current unemployment rate in Italy? \\
        A-Few-Months & Within a few months & What is the current inflation rate in the Eurozone? \\
        A-Year & Within a year & What is the current version of the Java programming language? \\
        A-Few-Years & Within a few years & Who is the president of the United States? \\
        Many-Years & After many years & What is the population of Germany? \\
        Never & Never changes & What is the chemical symbol for gold? \\
        \end{tabular}
    }
    \caption{The recency classes proposed by the \textit{RecencyQA} paper, along with example questions for each class.}
    \label{tab:recency-classes}
\end{table*}
\begin{itemize}
    \item \textbf{Stationarity}: Whether a questions \emph{recency requirement} is context-invariant. \emph{Stationary} questions keep the same recency label across contexts, whereas \emph{non-stationary} questions change their label when contextual conditions change.
    \item \textbf{Multi-Event}: Questions may depend on a single situation or combine information from multiple distinct or connected events. Multi-event questions increase temporal and structural complexity by requiring the integration of information across several concurrent or related processes.
    \item \textbf{Inter-event relationship}: When multiple events are involved, their relationship can be causal—where one event influences another—or purely temporal, where events co-occur in time without an assumed dependency.
    \item \textbf{Recency classes}: Each question is associated with a recency class indicating the expected timescale on which its answer may change (e.g., hours, days, or years). We adopt the recency classes proposed by the \textit{RecencyQA} paper (see Table~\ref{tab:recency-classes}).
    \item \textbf{Contextual conditions}: Recency is not absolute but depends on the assumed state of the world. Contextual conditions describe the situation under which a given recency class applies, allowing the same question (under Non-Stationary) to have different recency labels.
\end{itemize}

\subsection{Question Selection}
Having defined the question taxonomy, the next step in the pipeline is to select an initial set of seed questions from which new items are generated. These seeds serve as prompts that anchor topic, style, and temporal structure while still allowing the language model to introduce new events, domains, and cross-event interactions.

In our setting, we randomly select \num{75} questions from the original \textit{RecencyQA} dataset as seed inputs. The selected questions are evenly distributed across stationarity and recency classes. All questions are provided to the model in a structured \texttt{JSON} format (see Appendix~\ref{app:sec:input-json}).

\subsection{Question generation}
Given the set of \num{75} seed questions described in the previous section, the next stage of the pipeline expands each seed into a diverse collection of recency-aware variants that systematically explore the taxonomy introduced in Section~\ref{sec:question-taxonomy}. The goal is not to paraphrase the original questions, but to create new, semantically distinct questions that preserve similar structure while varying event composition and temporal behavior. As mentioned earlier, we utilize the \textit{Llama-3.3-70B-Instruct-Turbo}~\cite{touvron2023llamaopenefficientfoundation} model for this generation task.

\subsubsection{Generation Setup}
For each seed question, we invoke a family of prompt templates that condition the model on specific combinations of stationarity and event structure. These templates guide the model to generate questions that are either \emph{stationary} or \emph{non-stationary}, and that depend on either a \emph{single event} or on \emph{multiple} related events. For multi-event questions, we further distinguish between \emph{causal} relationships—where one event influences another—and \emph{temporal-only} relationships, where events merely co-occur in time without direct dependency. To increase task difficulty and better probe temporal reasoning capabilities (see Section~\ref{sec:results}), we additionally control the structural complexity by generating multi-event questions involving either two or three distinct events, requiring models to integrate information across multiple evolving processes.

\subsubsection{Prompt Design}
Each prompt template is designed to generate two questions per call and explicitly forbids placeholders or paraphrases of the seed, encouraging the generation of genuinely new content rather than surface-level reformulations (Appendix~\ref{app:gen-prompts}). In practice, however, the language model may very rarely return fewer than two valid questions or fail to produce a usable output. In such cases, only the successfully generated questions are retained.

\begin{table}[!t]
    \centering
    \begin{tabular}{cccc}
        \toprule
        \textbf{St.} & \textbf{\#Ev.} & \textbf{Rel.} & \textbf{Prompt} \\
        \midrule
        S  & 1 & -- & \ref{prompt:stat-single} \\
        S  & 2 & C  & \ref{prompt:stat-2-causal} \\
        S  & 2 & T  & \ref{prompt:stat-2-temporal} \\
        S  & 3 & C  & \ref{prompt:stat-3-causal} \\
        S  & 3 & T  & \ref{prompt:stat-3-temporal} \\
        \midrule
        NS & 1 & -- & \ref{prompt:nonstat-single} \\
        NS & 2 & C  & \ref{prompt:nonstat-2-causal} \\
        NS & 2 & T  & \ref{prompt:nonstat-2-temporal} \\
        NS & 3 & C  & \ref{prompt:nonstat-3-causal} \\
        NS & 3 & T  & \ref{prompt:nonstat-3-temporal} \\
        \bottomrule
    \end{tabular}
    \caption{Prompt families used for question generation. 
        \textbf{St.} = Stationarity (S = Stationary, NS = Non-stationary), 
        \textbf{\#Ev.} = number of events, 
        \textbf{Rel.} = inter-event relation (C = causal, T = temporal-only).
    }
    \label{tab:prompt-families}
\end{table}
Table~\ref{tab:prompt-families} summarizes the ten prompt families used to cover the full space of temporal structures defined by our taxonomy. Each seed question is processed by all ten prompt variants, resulting in up to twenty new questions per seed (two per prompt). Thus our complete generation process yields up to \num{1500} new questions, subject to filtering for malformed outputs as described below.

We provide, for each prompt template, representative example questions together with their accompanying contexts and the assigned recency label(s); see Appendix~\ref{app:example-questions-each-variation}.

\subsubsection{Annotation and Metadata}
\label{sec:annotation-and-metadata}
Every generated question is automatically associated with an identifier and annotated with metadata describing its stationarity, event dependency, number of events, and—when applicable—the prompted inter-event relation (causal vs. temporal-only). This process transforms each original question into a family of temporally enriched questions, yielding a dataset that supports fine-grained analysis of how temporal structure, multi-event dependency, and recency demand interact in LLM-based question answering.

\subsection{Recency and context labeling}
To apply the recency-class dimension introduced in Section~\ref{sec:question-taxonomy}, we annotate each question with one recency class (or two recency classes if non-stationary) that specify on which time scale its correct answer is expected to change. We use the twelve classes proposed by the \textit{RecencyQA} paper (Table~\ref{tab:recency-classes}) and store them as normalized strings (e.g., \texttt{An-Hour}, \texttt{A-Few-Days}).

Stationary questions receive a single label, while non-stationary questions receive two labels to capture alternative temporal regimes under different world states. For each label, we additionally elicit a one-sentence contextual condition (event, phase, or condition) that makes the question relevant under that regime. The labeling prompt prohibits reasoning and explicit timestamps and is reproduced in Appendix~\ref{app:label-prompts} via the stationary template (Appendix~\ref{prompt:label-stat}) and the non-stationary template (Appendix~\ref{prompt:label-nonstat}). Labels and conditions are merged into a structured list so downstream models can directly pair each recency class with its motivating scenario. We retain only entries whose labels parse as \texttt{JSON} and whose number of conditions matches the expected label count.

The output is stored in a structured \texttt{JSON} format (see Appendix~\ref{app:sec:output-json}) that includes the questions, recency labels, contextual conditions, and the metadata described (Section~\ref{sec:annotation-and-metadata}). In Appendix~\ref{app:example-questions-json}, we provide a concrete example of a question with its associated labels and contexts in \texttt{JSON} format.

\subsection{Verification and quality control}
To reduce noise from imperfect generations and labels, we apply a two-stage verification procedure combining an LLM-based consistency check with targeted manual review.

First, we used \textit{GPT-5.1 Codex Max} via GitHub Copilot Premium to screen all records. To fit the model's context window, we split the dataset into seven parts and asked the model to validate (i) whether the recency label(s) match the question under the provided contextual condition(s) and (ii) whether the question is sensible and grammatically well-formed, see Appendix~\ref{sec:verification-prompt} for the exact prompt. The model outputs a list of question IDs flagged as either \enquote{correct} or \enquote{incorrect}.

All items flagged as \enquote{incorrect} were then double-checked by hand and removed.
Finally, we randomly sampled \num{75} remaining questions and audited them manually; only rare edits were necessary (e.g., adjusting a label or removing an ambiguous question).

\subsection{Dataset Statistics}
We conclude this section by summarizing the resulting dataset after generation, labeling, and verification. Table~\ref{tab:dataset-stats} provides an overview of the key corpus characteristics that we use throughout our experiments and analysis. Figure~\ref{fig:recency-label-distribution} visualizes the distribution of recency labels across the entire dataset.

% Total questions (all splits) & 1411 \\
% Unique seed questions & 1373 \\
% Stationary questions & 725 \\
% Non-stationary questions & 686 \\
% Single-event questions & 286 \\
% Two-event questions & 555 \\
% Three-event questions & 570 \\
% Multi-event (causal) & 565 \\
% Multi-event (temporal-only) & 560 \\
% 1-label items & 725 \\
% 2-label items & 686 \\
% Avg. question length (words) & 22.2 \\
% Avg. context 1 length (words) & 11.2 \\
% Avg. context 2 length (words) & 12.6 \\
% recency_label1 count & 1400 \\
% recency_label2 count & 697 \\
% Most frequent labels (top-3) & [('A-Year', 373), ('A-Day', 352), ('A-Few-Days', 320)] \\
\begin{table}[!t]
    \centering
    \small
    \renewcommand{\arraystretch}{1.2}
    \setlength{\tabcolsep}{4pt}
    \begin{tabular}{l@{\hspace{0.8em}}r}
        \toprule
        \textbf{Metric} & \textbf{Value} \\
        \midrule
        Total questions (all splits) & \num{1411} \\
        Unique seed questions & \num{1373} \\
        Total Recency Classes & \num{12} \\
        \midrule
        Stationary questions & \num{725} \\
        Non-stationary questions & \num{686} \\
        \midrule
        Single-event questions & \num{286} \\
        Two-event questions & \num{555} \\
        Three-event questions & \num{570} \\
        Multi-event (causal) & \num{565} \\
        Multi-event (temporal-only) & \num{560} \\
        \midrule
        Avg. question length (tokens) & \num{22.2} \\
        Avg. context 1 length (tokens) & \num{11.2} \\
        Avg. context 2 length (tokens) & \num{12.6} \\
        \midrule
        Total recency labels & \num{2097} \\
        Most frequent labels (top-3) & \parbox[t]{0.35\linewidth}{\raggedright
            \texttt{A-Year} (\num{373}),\\
            \texttt{A-Day} (\num{352}),\\
            \texttt{A-Few-Days} (\num{320})
        } \\
        \bottomrule
    \end{tabular}
    \caption{Dataset overview statistics.}
    \label{tab:dataset-stats}
\end{table}

\begin{figure}[t]
    \centering
    \includegraphics[width=0.48\textwidth]{res/recency_labels_treemap.pdf}
    \caption{Distribution of recency labels in the final dataset.}
    \label{fig:recency-label-distribution}
\end{figure}

\section{Testing Pipeline}
\begin{frame}
  \frametitle{Testing pipeline}
  \begin{itemize}
    \item Flatten dataset: each question-context-label pair becomes one instance.
    \item Testing on three models:
    \begin{itemize}
      \item \emph{Kimi-K2-Instruct-0905}
      \item \emph{Qwen2.5-72B-Instruct-Turbo}
      \item \emph{DeepSeek-V3}
    \end{itemize}
    \item Collect predictions per model into \texttt{JSONL}
    \item Summarize accuracy and tolerant ($\pm 1$ label) accuracy.
    \item Slice metrics:
    \begin{itemize}
      \item Stationary vs. non-stationary
      \item Single-event vs. multi-event
      \item Causal vs. temporal-only
    \end{itemize}
  \end{itemize}
\end{frame}


\section{Results}
Building on the question taxonomy from Section~\ref{sec:question-taxonomy} and the metadata annotations discussed in Section~\ref{sec:annotation-and-metadata}, we evaluate \textit{moonshotai/Kimi-K2-Instruct-0905} (\num{1} Trillion parameters)~\cite{moonshotai_Kimi-K2-Instruct-0905_2025}, \textit{Qwen/Qwen2.5-72B-Instruct-Turbo} (\num{72} Billion parameters)~\cite{qwen2.5,qwen2} and \textit{deepseek-ai/DeepSeek-V3} (\num{671} Billion parameters)~\cite{deepseekai2025deepseekv3technicalreport} with the testing pipeline from Section~\ref{sec:testing-pipeline}. 

\begin{table*}[t]
	\centering
	\small
	\setlength{\tabcolsep}{6pt}
	\begin{tabular}{l|cc|cc|cc|ccc|cc}
		\toprule
		& \multicolumn{2}{c|}{Overall} & \multicolumn{2}{c|}{St.} & \multicolumn{2}{c|}{Non-St.} & \multicolumn{3}{c|}{\# Events} & \multicolumn{2}{c}{Multi-event} \\
		Model & Acc & Tol. & Acc & Tol. & Acc & Tol. & 1 Acc & 2 Acc & 3 Acc & C Acc & T Acc \\
		\midrule
		Kimi & \num{33.0} & \num{62.2} & \num{42.1} & \num{66.3} & \num{28.3} & \num{60.0} & \num{35.0} & \num{31.1} & \num{33.6} & \num{34.9} & \num{29.8} \\
		Qwen & \num{45.6} & \num{77.4} & \num{62.8} & \num{81.8} & \num{36.7} & \num{75.1} & \num{45.4} & \num{44.6} & \num{46.6} & \num{47.8} & \num{43.5} \\
		Deepseek & \num{40.8} & \num{71.9} & \num{52.2} & \num{73.2} & \num{34.9} & \num{71.1} & \num{44.9} & \num{38.1} & \num{41.1} & \num{43.9} & \num{35.6} \\
		\midrule
		Total & \num{39.8} & \num{70.5} & \num{52.4} & \num{73.8} & \num{33.3} & \num{68.7} & \num{41.7} & \num{37.9} & \num{40.5} & \num{42.2} & \num{36.3} \\
		\bottomrule
	\end{tabular}
	\caption{Accuracy (Acc) and tolerant F1 accuracy (Tol., $\pm 1$ label; computed as the tolerant F1 metric). St.: stationary; Non-St.: non-stationary; C: causal; T: temporal-only.}
	\label{tab:model-overview}
\end{table*}

\subsection{Overall Performance}
Qwen2.5-72B attains the strongest accuracy (\num{45.6}\unit{\%}) and tolerant F1 accuracy (\num{77.4}\unit{\%}). DeepSeek-V3 trails by roughly five absolute points, while the Kimi-K2 instruction model stays below \num{35}\unit{\%} accuracy despite a comparable tolerant score. Invalid generations remain negligible, confirming that the constrained testing prompt in Appendix~\ref{app:testing-prompts} keeps responses well-formed.

\subsection{Impact of Stationarity}
Stationarity is the clearest driver of variance. Each model loses between \num{13} and \num{26} percentage points when moving from stationary to non-stationary questions, mirroring the temporal volatility highlighted during annotation (Section~\ref{sec:annotation-and-metadata}). Qwen drops from \num{62.8}\% to \num{36.7}\% accuracy, indicating that even large instruction models struggle when the required recency hinges on punctual events rather than cyclical updates. Kimi suffers the steepest relative decline ($-13.8$ points) because its stationary accuracy is already modest, whereas DeepSeek maintains low-\num{30}s performance on non-stationary prompts despite exceeding \num{52}\% on the stable slice. These gaps confirm that the dataset captures the adaptive reasoning behaviour targeted in Section~\ref{sec:question-taxonomy}.

\subsection{Mutli-Event Question Performance}
In the final four columns of Table~\ref{tab:model-overview}, we break down model performance on multi-event questions by number of events and generation type: causal (C) vs. temporal-only (T). For all three models we don't observe significant performance differences based on the number of events involved in the question. The tolerant accuracy has a slightly more downward trend as the number of events increases (from \num{80.3}\unit{\%} to \num{66.2}\unit{\%} (two events) and \num{69.3}\unit{\%} (three events) (aggregated)). This does indicate that multi-event questions are more difficult than single-event questions, but the number of events itself does not seem to have a strong influence. 

When comparing causal vs. temporal-only multi-event questions, we observe that all models perform better on causal questions. Qwen achieves \num{47.8}\unit{\%} accuracy on causal questions compared to \num{43.5}\unit{\%} on temporal-only questions. Kimi and DeepSeek show similar trends with \num{34.9}\unit{\%} vs. \num{29.8}\unit{\%} and \num{43.9}\unit{\%} vs. \num{35.6}\unit{\%}, respectively. This suggests that models may find it easier to reason about cause-and-effect relationships when determining recency, as opposed to purely temporal relationships. Though the differences are not very large, they are consistent across all models.


\begin{figure}
  \centering
  \includegraphics[width=\linewidth]{res/aggregated_confusion_heatmap.pdf}
  \caption{Model-averaged confusion matrix over all 12 recency labels, showing recall per label.}
  \label{fig:recency-confusion}
\end{figure}
\subsection{Label-wise Behaviour and Error Direction}
Per-label statistics expose systematic blind spots. All three systems excel at the shortest horizons (e.g., Qwen reaches recall \num{55}\unit{\%} on \texttt{A-Few-Hours} and Kimi exceeds \num{73}\unit{\%}), yet accuracy collapses for intermediate windows: \texttt{A-Week} recall never surpasses \num{10}\unit{\%}, and \texttt{A-Month} remains below \num{20}\unit{\%} for Kimi and DeepSeek. Long-horizon labels such as \texttt{Many-Years} and \texttt{Never} also exhibit low precision (Qwen's best F$_1$ for \texttt{Never} is \num{6.5}\unit{\%}). A condensed, model-averaged confusion heatmap makes these failure modes visible; we therefore refer to Figure~\ref{fig:recency-confusion} for the aggregated confusion matrix over all 12 labels.

\begin{figure}
  \centering
  \includegraphics[width=\linewidth]{res/error_direction_barplot.pdf}
  \caption{Directional error counts per model, contrasting \enquote{too early} vs. \enquote{too late} predictions.}
  \label{fig:error-direction}
\end{figure}

Directional errors emphasise another imbalance. Kimi produces \num{801} \enquote{too early} predictions versus \num{591} \enquote{too late}, Qwen \num{695} vs. \num{439}, and DeepSeek \num{652} vs. \num{582}. The models thus prefer overly fresh information, underestimating how slowly certain questions evolve. This bias matters for downstream systems that rely on recency signals to schedule refreshes: adopting a \enquote{check too often} policy could waste computation, whereas missing late updates risks outdated answers. To visualize this effect we refer to Figure~\ref{fig:error-direction}, a bar chart contrasting the error directions per model.

Finally, tolerant accuracy being between \num{20} and \num{30} points higher than accuracy across all models shows that most mistakes deviate by only one label. While encouraging, this plateau also signals that the discrete label borders defined in Section~\ref{sec:question-taxonomy} remain hard to recover without explicit temporal reasoning, motivating future work on richer reasoning prompts or retrieval-augmented pipelines.


\section{Fine-Tuning}
\begin{frame}
  \frametitle{Fine-tuning setup}
  \begin{columns}
    \begin{column}{0.37\textwidth}
      \begin{itemize}
      \item Smaller Qwen model (\num{14}B)
      \item Split Dataset \num{70}/\num{15}/\num{15} 
      \item Finetuned via Together AI
      \item Compared to previous models
    \end{itemize}
    \end{column}
    \begin{column}{0.63\textwidth}
      \vspace{-2em}
      \begin{figure}[h]
        \centering
        \includegraphics[width=\linewidth]{_images/FineTuneOverview.png}
      \end{figure}
    \end{column}
  \end{columns}
\end{frame}

\begin{frame}[c]
  \frametitle{Fine-tuning results (reduced test set)}
  \scriptsize
  \centering
  \resizebox{\textwidth}{!}{%
  \begin{tabular}{l|cc|cc|cc|cc|cc}
    \toprule
    & \multicolumn{2}{c|}{Overall} & \multicolumn{2}{c|}{St.} & \multicolumn{2}{c|}{Non-St.} & \multicolumn{2}{c|}{Single-event} & \multicolumn{2}{c}{Multi-event} \\
    Model & Acc & Tol. & Acc & Tol. & Acc & Tol. & Acc & Tol. & Acc & Tol. \\
    \midrule
    Qwen2.5-14B (FT) & \num{40.9} & \num{69.6} & \num{55.1} & \num{72.9} & \num{33.5} & \num{68.0} & \num{41.5} & \num{78.5} & \num{40.7} & \num{67.3} \\
    Kimi & \num{35.5} & \num{64.9} & \num{46.3} & \num{70.4} & \num{29.8} & \num{62.0} & \num{29.7} & \num{65.6} & \num{36.9} & \num{64.7} \\
    Qwen2.5-72B & \num{46.5} & \num{76.1} & \num{63.9} & \num{82.4} & \num{37.4} & \num{72.8} & \num{46.2} & \num{87.7} & \num{46.6} & \num{73.1} \\
    DeepSeek-V3 & \num{41.4} & \num{69.7} & \num{50.9} & \num{71.3} & \num{36.4} & \num{68.9} & \num{38.5} & \num{70.8} & \num{42.2} & \num{69.5} \\
    \bottomrule
  \end{tabular}%
  }
\end{frame}

% \begin{frame}[c]
%   \frametitle{Fine-tuning results — accuracy barplot}
%   \centering
%   \includegraphics[width=0.85\linewidth]{../Paper/res/finetuned_accuracy_barplot.pdf}
% \end{frame}

\begin{frame}[plain]
  \frametitle{Parameter efficiency}
  \centering
  \includegraphics[width=0.85\linewidth]{../Paper/res/finetuned_param_normalized.pdf}
\end{frame}


\section{Conclusion and Outlook}
Having detailed the dataset construction, evaluation, and fine-tuning analysis, we now synthesize the main takeaways.

\subsection{Conclusion}
We introduced \textit{RecencyQA-Multi}, a systematically generated extension of the original RecencyQA corpus that explicitly varies stationarity, the number of interdependent events, and the relationships between them (Section~\ref{sec:dataset-creation}). Starting from only \num{75} seed questions, our controlled prompting and verification pipeline yielded \num{1411} questions with fine-grained recency labels and contextual conditions, providing a richer testbed for studying temporal reasoning.

To gauge how well current models exploit this structure, we designed a reusable evaluation pipeline (Section~\ref{sec:testing-pipeline}) and benchmarked state-of-the-art instruction models (Section~\ref{sec:results}). Despite tolerant accuracies exceeding \num{70}\unit{\%}, all models misclassify most instances under the strict metric, and accuracy drops sharply for non-stationary and multi-event questions. Qwen2.5-72B performs best overall, yet still struggles to distinguish medium-horizon labels and consistently predicts overly fresh answers, underscoring the difficulty of calibrating temporal priors even for large LLMs.

Fine-tuning a smaller model confirms that adaptation can close much of the scale gap: the \textit{Qwen2.5-14B-Instruct-recency} model reaches \num{40.9}\unit{\%} accuracy and \num{69.6}\unit{\%} tolerant F$_1$ on the reduced test set, matching DeepSeek-V3 and surpassing Kimi-K2 while trailing Qwen2.5-72B. When normalized by parameter count, the fine-tuned model delivers the strongest performance per parameter (Figure~\ref{fig:finetune-param-efficiency}), underscoring that targeted fine-tuning yields substantial efficiency gains.

\subsection{Future Research}
Several directions for future work could strengthen both the reliability and expressiveness of recency labels. A natural next step is to construct a dataset in which labels are assigned by humans at scale, ideally with multiple independent annotators per question. This could be facilitated through a lightweight web interface, and the final label could be derived from an aggregation rule such as the median or majority vote to reduce individual bias and noise. This would allow a more accurate evaluation of model performance against human judgment.

In addition, model ensembles offer a complementary avenue: multiple models could be run in parallel and the most consistently predicted label could be used as the final decision. In cases where the ensemble remains inconclusive, additional models (or a second-stage decision process) could be introduced, informed by the earlier model outputs to guide disambiguation. Potentially, this could yield more robust recency estimates by leveraging diverse model perspectives.

Finally, instead of treating recency prediction as a single hard-label task, it may be more appropriate to model ambiguity explicitly by using a distribution over labels as the target. For example, if annotators disagree between \enquote{A-Week} and \enquote{A-Month}, the training signal could reflect this uncertainty via proportional target probabilities. Such soft targets would be less brittle than fixed labels and could better capture the inherently fuzzy boundary between time ranges.


\begin{frame}
  
\end{frame}

\begin{frame}
  \frametitle{Human labeling}
  \centering
  \vfill
  \includegraphics[width=0.3\linewidth]{_images/qrcode.pdf}
  \vfill
  \url{https://recency-labeling-page.vercel.app/}
\end{frame}
\begin{frame}
  \frametitle{Labeling and verification}
  \begin{itemize}
    \item Label recency classes: 
    \begin{itemize}
      \item Stationary: single label + context
      \item Non-Stationary: two labels + contexts 
    \end{itemize}
    \item Use \emph{Llama-3.3-70B-Instruct-Turbo} for labeling
    \item Enforce structured JSON
    \item LLM screening (\emph{GPT-5.1 Codex Max}) for label-question consistency
    \item Manual audit of flagged items; remove or fix ambiguous cases
    \begin{itemize}
      \item Sample \num{75} items for manual verification
    \end{itemize}
  \end{itemize}
\end{frame}

\end{document}

